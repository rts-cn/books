\documentclass[$if(fontsize)$$fontsize$,$endif$$if(lang)$$lang$,$endif$CJKutf8]{$documentclass$}
\XeTeXlinebreaklocale "zh"
\XeTeXlinebreakskip = 0pt plus 1pt minus 0.1pt
\usepackage[top=1in,bottom=1in,left=1.25in,right=1.25in]{geometry}
\usepackage{float}
\usepackage{fontspec}
\newfontfamily\zhfont[BoldFont=Adobe Heiti Std]{Adobe Song Std}
\newfontfamily\zhpunctfont{Adobe Song Std}
\setmainfont{Times New Roman}
\usepackage{indentfirst}
\usepackage{zhspacing}
\zhspacing

% \usepackage[papersize={19cm, 23.6cm},text={15cm, 18.6cm}]{geometry}
% \usepackage[text={15cm, 18.6cm}]{geometry}
% \addtolength{\oddsidemargin}{-0.6cm}
% \addtolength{\evensidemargin}{0.4cm}

\usepackage{longtable}
\usepackage{indentfirst}
\setlength{\parindent}{2em}
% \renewcommand\thechapter{}
\usepackage{fancyvrb}
%%%%%%%%%%%%%%%%%%%%%%%%%%%%%%%%%%%%%%%%%%%%%%%%%%%%%%%%
% 定义了自己的verbatim环境,并且简写为myv
%%%%%%%%%%%%%%%%%%%%%%%%%%%%%%%%%%%%%%%%%%%%%%%%%%%%%%%%
\DefineVerbatimEnvironment{verbatim}{Verbatim}
{frame=none,
baselinestretch=1,
fontsize=\small,
xleftmargin=5pt,
xrightmargin=5pt,
framesep=5mm,
numbers=left,
samepage=true
}

% \usepackage{perpage} %the perpage package
% \MakePerPage{footnote} %the perpage package command

\usepackage{fancyhdr}
\pagestyle{fancy}
\fancyhead{}
\fancyfoot{}
\fancyhead[CO,CE]{FreeSWITCH权威指南(附录)}
\fancyfoot[C]{------------ 版权所有,侵权必究 -----------}
% \fancyfoot[C]{------------ D R A F T -- Please keep this secret!! ----------}
% \fancyfoot[C]{------------ 草稿\quad保密\quad保留所有权利------------}
\fancyfoot[RO, LE] {\thepage}
\fancyhead[LE,RO]{\chaptermark}
\fancyhead[LO,RE]{\sectionmark}

\renewcommand{\contentsname}{目\quad 录}
\renewcommand\listfigurename{插图目录}
\renewcommand\listtablename{表格目录}
% \renewcommand\refname{参考文献}
\renewcommand\indexname{索引}
\renewcommand\figurename{图}
\renewcommand\tablename{表}
\renewcommand\abstractname{摘要}
\renewcommand\partname{部分}
\renewcommand\appendixname{附录}
\renewcommand\today{\number\year年\number\month月\number\day日}
\providecommand{\CJKnumber}[1]{\ifcase#1\or{一}\or{二}\or{三}\or{四}\or{五}\or{六}\or{七}\or{八}\or{九}\or{十}\or{十一}\or{十二}\or{十三}\or{十四}\or{十五}\or{十六}\or{十七}\or{十八}\or{十九}\or{二十}\or{二十一}\or{二十二}\or{二十三}\or{二十四}\or{二十五}\or{二十六}\or{二十七}\or{二十八}\or{二十九}\or{三十}\fi}

% \renewcommand{\thesection}{\Alph{section}}

\setcounter{chapter}{4}
\usepackage{titlesec}
\titleformat{\chapter}{\Huge\bfseries}{附录\,\Alph{chapter}\,}{1em}{}
\renewcommand{\thechapter}{\Alph{chapter}}

\usepackage{fancyvrb}
\fvset{fontsize=\footnotesize}
\fvset{xleftmargin=0.8cm}
\fvset{frame=lines,framerule=1pt}
\RecustomVerbatimEnvironment{verbatim}{Verbatim}{}

\usepackage{lmodern}
\usepackage{amssymb,amsmath}
\usepackage{ifxetex,ifluatex}
\usepackage{fixltx2e} % provides \textsubscript
% use microtype if available
\IfFileExists{microtype.sty}{\usepackage{microtype}}{}
\ifnum 0\ifxetex 1\fi\ifluatex 1\fi=0 % if pdftex
  \usepackage[utf8]{inputenc}
$if(euro)$
  \usepackage{eurosym}
$endif$
\else % if luatex or xelatex
  \usepackage{fontspec}
  \ifxetex
    \usepackage{xltxtra,xunicode}
  \fi
  \defaultfontfeatures{Mapping=tex-text,Scale=MatchLowercase}
  \newcommand{\euro}{€}
$if(mainfont)$
    \setmainfont{$mainfont$}
$endif$
$if(sansfont)$
    \setsansfont{$sansfont$}
$endif$
$if(monofont)$
    \setmonofont{$monofont$}
$endif$
$if(mathfont)$
    \setmathfont{$mathfont$}
$endif$
\fi
$if(geometry)$
\usepackage[$for(geometry)$$geometry$$sep$,$endfor$]{geometry}
$endif$
$if(natbib)$
\usepackage{natbib}
\bibliographystyle{plainnat}
$endif$
$if(biblatex)$
\usepackage{biblatex}
$if(biblio-files)$
\bibliography{$biblio-files$}
$endif$
$endif$
$if(listings)$
\usepackage{listings}
$endif$
$if(lhs)$
\lstnewenvironment{code}{\lstset{language=Haskell,basicstyle=\small\ttfamily}}{}
$endif$
$if(highlighting-macros)$
$highlighting-macros$
$endif$
$if(verbatim-in-note)$
\usepackage{fancyvrb}
$endif$
$if(fancy-enums)$
% Redefine labelwidth for lists; otherwise, the enumerate package will cause
% markers to extend beyond the left margin.
\makeatletter\AtBeginDocument{%
  \renewcommand{\@listi}
    {\setlength{\labelwidth}{4em}}
}\makeatother
\usepackage{enumerate}
$endif$
$if(tables)$
\usepackage{ctable}
\usepackage{float} % provides the H option for float placement
$endif$
$if(graphics)$
\usepackage{graphicx}
% We will generate all images so they have a width \maxwidth. This means
% that they will get their normal width if they fit onto the page, but
% are scaled down if they would overflow the margins.
\makeatletter
\def\maxwidth{\ifdim\Gin@nat@width>\linewidth\linewidth
\else\Gin@nat@width\fi}
\makeatother
\let\Oldincludegraphics\includegraphics
\renewcommand{\includegraphics}[1]{\Oldincludegraphics[width=\maxwidth]{#1}}
$endif$
\ifxetex
  \usepackage[setpagesize=false, % page size defined by xetex
              unicode=false, % unicode breaks when used with xetex
              xetex]{hyperref}
\else
  \usepackage[unicode=true]{hyperref}
\fi
\hypersetup{breaklinks=true,
            bookmarks=true,
            pdfauthor={$author-meta$},
            pdftitle={$title-meta$},
            colorlinks=true,
            urlcolor=$if(urlcolor)$$urlcolor$$else$blue$endif$,
            linkcolor=$if(linkcolor)$$linkcolor$$else$magenta$endif$,
            pdfborder={0 0 0}}
$if(links-as-notes)$
% Make links footnotes instead of hotlinks:
\renewcommand{\href}[2]{#2\footnote{\url{#1}}}
$endif$
$if(strikeout)$
\usepackage[normalem]{ulem}
% avoid problems with \sout in headers with hyperref:
\pdfstringdefDisableCommands{\renewcommand{\sout}{}}
$endif$
\setlength{\parindent}{2em}
\setlength{\parskip}{6pt plus 2pt minus 1pt}
\setlength{\emergencystretch}{3em}  % prevent overfull lines
$if(numbersections)$
$else$
\setcounter{secnumdepth}{0}
$endif$
$if(verbatim-in-note)$
\VerbatimFootnotes % allows verbatim text in footnotes
$endif$
$if(lang)$
\ifxetex
  \usepackage{polyglossia}
  \setmainlanguage{$mainlang$}
\else
  \usepackage[$lang$]{babel}
\fi
$endif$
$for(header-includes)$
$header-includes$
$endfor$

$if(title)$
\title{$title$}
$endif$
\author{$for(author)$$author$$sep$ \and $endfor$}
\date{$date$}

\begin{document}
% \begin{CJK}{UTF8}{gbsn}

%% start ncip
\newcommand{\thetitle}{FreeSWITCH权威指南(附录)}
\newcommand{\theauthor}{杜金房}
\newcommand{\theauthors}{杜金房 \quad 张令考}
\newcommand{\thepublisher}{版权所有,侵权必究}


\thispagestyle{empty}

\begin{center}
	{\Huge \bf \thetitle\\[1em]}
  {\Large\bf \theauthors \quad\\[2em]}
\end{center}

\vfill
\begin{center}
	\Large{\bf \thepublisher}
\end{center}


\newpage

\chapter*{}

本文档是《FreeSWITCH 权威指南》(ISBN 978-7-111-46626-0)一书的电子版附录。由于原书 太“厚”了,因此单独将这一部分以电子版形式发布。

本文档版权完全归作者所有,仅供购买了《FreeSWITCH 权威指 南》实体书的读者使用,严禁任何形式的侵权行为!

本书网站:\href{http://book.dujinfang.com}{book.dujinfang.com} 。

\newpage

$if(title)$
\maketitle
$endif$

$for(include-before)$
$include-before$

$endfor$
$if(toc)$
{
\hypersetup{linkcolor=blue}
\tableofcontents
}
$endif$
$body$

\newpage
\chapter*{写在最后}

在笔者及出版社编辑和工作人员的共同努力下,本书终于要和广大读者见面了。欣慰之余,也算是给经历了漫长的等待时间的读者一个交代吧。

本书的写作时间跨度较长,而FreeSWITCH又更新太快,以至于有些章节都先后修改了好几遍。尤其是FreeSWITCH自去年起分成了两个分支,兼顾两个分支无疑大大增加了写作的难度。本书截稿后到出版前的这段时间,FreeSWITCH又有了很大的变化,使我们多少有些被动。不过,令人高兴的是,FreeSWITCH在三天前发布了1.4.4版---这是FreeSWITCH 1.4自去年发布以来的第一个正式版!同时,这也意味着,FreeSWITCH自1.2版以来巨大变化的节奏终于可以消停一会儿了。这也使得笔者在本书即将面市之即,还有机会在电子版附录里写点东西。

FreeSWITCH现在有两个版本---1.2版和1.4版。它们之间的最大区别就是后者支持WebRTC。为了支持WebRTC,FreeSWITCH核心代码也进行了一系列的重构,这些内容都已经涵盖在本书里了。而在本书截稿之后FreeSWITCH中最大的变化就是对FS-353(\href{http://jira.freeswitch.org/browse/FS-353}{jira.freeswitch.org/browse/FS-353})的支持。事情还得从多年前说起---在FreeSWITCH中,为了避免重复发明轮子,使用了很多的第三方库。最初大部分第三方库的代码也放在FreeSWITCH的代码库中,静态编译连接,这样非常易于从源代码进行编译;而FS-353指出,这种方式不利于将FreeSWITCH打包加入各发行版(典型的如不同的Linux发行套件),因而最新的FreeSWITCH不再使用自己代码库中的第三方代码,而是更多地依赖于操作系统提供的第三方库。现在,如果需要编译最新的FreeSWITCH代码,除了按第3章描述的安装依赖库外,还需要安装更多的依赖库。当然,它也会带来很多好处,其中最重要的一点就是FreeSWITCH有望进入各发行版,而普通人员将再也不需要从源代码编译安装,而是可以直接使用发行版提供的工具如apt-get或yum等来安装FreeSWITCH了。关于这一点更详细的解释请参考我专门写的文章(\href{http://zhuanlan.zhihu.com/freeswitch/19746509}{zhuanlan.zhihu.com/freeswitch/19746509}),在此,就不赘述了。

在FreeSWITCH1.2版开发时,大部分是基于CentOS 5的,而随着时间的推移,CentOS 5已经老去(官方已停止支持),而新的CentOS 6一直有一些性能问题(可能到现在已经解决了),因而FreeSWITCH的开发者们在开发1.4版时都纷纷转向了Debian 7(同样老去的还有Debian 6)。所以,1.2版适于CentOS 5及Debian 6,而1.4版本最好是在Debian 7或CentOS 6上运行\footnote{当然,这并不是说1.4版不能在CentOS 5及Debian 6上运行,只是,需要费点劲,参见\href{https://confluence.freeswitch.org/display/FREESWITCH/CentOS+5}{https://confluence.freeswitch.org/display/FREESWITCH/CentOS+5} 。}。当然,除Linux外,1.4版还支持Windows、Mac OS X、BSD系列、Solaris等多种操作系统。

在本书写作时,就最大程度地兼顾了1.2版和1.4版。除了上面讲的几点老需要注意外,本书讲的内容都适合两个版本\footnote{当然,除了有些特性(如WebRTC)仅在1.4版上存在。}。因而,本书的内容在相当长的时间内都不会过时。当然,如果读者读完本书,对FreeSWITCH就应该了如只掌了,那时候如果再遇到什么问题,也都可以很容易地解决了。

除代码外,FreeSWITCH的文档系统也将由\href{http://wiki.freeswitch.org}{wiki.freeswithc.org}切换到\href{http://confluence.freeswitch.org}{confluence.freeswitch.org},后者将提供更好的SSO(单点登录)以及更专业的文档,但前者也将在一定时期内以只读的形式继续存在,以便读者参考。

无论如何,本书的在线站点(\href{http://book.dujinfang.com}{book.dujinfang.com})都会有本书最新的更新以及勘误等,也欢迎广大读者订阅微信公众社区“FreeSWITCH-CN”获取更多更及时的信息。

\bigskip
\vfill

                             杜金房/2014年5月26日于烟台

\chapter*{}


THIS PAGE INTENTIONALLY LEFT BLANK.

$if(natbib)$
$if(biblio-files)$
$if(biblio-title)$
$if(book-class)$
\renewcommand\bibname{$biblio-title$}
$else$
\renewcommand\refname{$biblio-title$}
$endif$
$endif$
\bibliography{$biblio-files$}

$endif$
$endif$
$if(biblatex)$
\printbibliography$if(biblio-title)$[title=$biblio-title$]$endif$

$endif$

$for(include-after)$
$include-after$

$endfor$


% \end{CJK}
\end{document}
